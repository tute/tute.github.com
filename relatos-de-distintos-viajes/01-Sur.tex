\section{Primeros d\'ias, lago Huechulaufquen}

Estoy en San Mart\'in de los Andes, y les cuento que estoy de diez. Esto es
incre\'ible, no lo puedo creer. El primer d\'ia embarcamos en las bicis hacia el
lago Huechulaufquen. \textexclamdown Eran las 2am y segu\'iamos pedaleando,
desde las 6:30pm! Nos agarr\'o la noche como a los mejores, pero seguimos a pie,
a la luz de la luna llena, que iluminaba la enormidad del lago. A las tres de la
ma\~nana paramos en el primer campamento que se nos cruz\'o, y empezamos a armar
la carpa y el calentador para cenar unos ricos capeletinis de verdura. Comimos,
ordenamos, y a la bolsa. A las 10:30 nos levantamos, fuimos al lago a lavar las
cosas, ordenamos todo, y partimos hacia el lago Paim\'un, muertos despu\'es de
las cinco horas de caminata/pedaleo de ayer con peso y monta\~na.
\textexclamdown Adem\'as, este primer d\'ia nos toc\'o viento en contra!

Hacia el Paim\'un tuvimos unas bajadas fenomenales. Con todo el peso de las
alforjas sobre la rueda de atr\'as, la delantera bailaba. A 25~km/h parec\'ian
240~km/h; y no nos lastimamos, lo que no quita que no nos hayamos ca\'ido.
Llegu\'e a 47~km/h y Ezequiel iba m\'as r\'apido. Un entretenido peligro en esas
bajadas-subidas, curvas-contracurvas, etc-etc.

Perdidos pas\'abamos el Paim\'un, cuando nos encontramos con Alfon --mi primo--
de pura casualidad. \textexclamdown Llegaban justo de cortar le\~na! Volvimos a
su campamento, donde nos esperaban los cuatro amigos con una trucha (que no fue
pescada por ellos, \textexclamdown aunque intentaron!) y algunos choclitos
hervidos. Luego caminamos a una cascada, que pudimos ver desde abajo y desde
atr\'as. \textexclamdown Parec\'ia una catarata y ten\'ia s\'olo 5 metros! De
los 45 minutos que tomaba llegar, 15' eran trepando una mota\~na a trav\'es de
rocas. Inolvidable.

Volvimos a ver c\'omo pescaban con mosca, un embole, pero al lado del lago y a
los pies de los cerros y del volc\'an Lan\'in, no la pasamos muy mal que
digamos. Dormimos ah\'i, y salimos la siguiente ma\~nana en un cole a San
Mart\'in de los Andes.

Aqu\'i estoy, despu\'es de dar unas vueltas para reconocer terreno. Fuimos a una
farmacia para ver qu\'e le pasa a Eze, y aunque le digan que no pasa nada no se
puede ni mover. No s\'e si sigue, vuelve o se queda, pero ma\~nana me gustar\'ia
seguir, lo vemos esta noche.

Un fuerte abrazo a todos, estoy de diez ac\'a en el Sur, conociendo. Bueno, nada
m\'as, muchas gracias, viejos;

Tute.

\section{Por la ruta de los 7 Lagos}

Eze se sinti\'o mejor, as\'i que de San Mart\'in de los Andes salimos hacia el
Lago Hermoso. Fue un poco duro pero estuvo mortal. Pasamos por el lago L\'acar
(donde compr\'e el pasaje, llego a Mar del Plata el 21 a la noche,
\textexclamdown espero carne con tomates!), el Mach\'onico (se ve\'ia lindo;
ah\'i habl\'e con un propietario de un {\small VW} Passat, \textexclamdown un
mentiroso sobre su auto!); y llegamos al lago Hermoso, donde acampamos y la
pasamos fr\'io. Venimos corriendo carreras porque ya estamos en Villa la
Angostura y pasado ma\~nana en ``Baril\'o'', planeado para dentro de una semana.

Del Hermoso salimos al Espejo Chico, pasando por el Falkner (de los m\'as
lindos), Villarino, y otros que ya no recuerdo. \textexclamdown Cierto, el
Escondido! Divino. Ten\'iamos tres capas de transpiraci\'on con sus respectivas
tres capas de tierra pegadas. Asquerosamente divertido.

Dormimos en el Espejo Chico, hermos\'isimo lago. Al fondo se empiezan a ver los
picos nevados de la Cordillera, m\'as cerca de la costa se ven truchas saltando,
y hay flores amarillas por todo el pasto. Un espect\'aculo. Comimos capeletinis
con ``pur\'e Cica'' arriba, riqu\'isimo.

De all\'i nos fugamos para Villa la Angostura. Pasamos por otros lagos, nos
cansamos mucho; por ese ripio con piedras a 25~km/h se sent\'ia como a 80~km/h.
Pero la cosa es que no \'ibamos a 25, \textexclamdown llegamos a 45~km/h con
alforjas y todo! Arribamos sanos y salvos y vivos, una bajada inolvidable. De
verdad. Cubiertos por un t\'unel de \'arboles y \'arboles que nos hac\'ian
sombra, en bicis sobre el ripio\ldots\ es un sue\~no que se hace realidad.
Pasamos por Pichi Traful, un camping muy organizado pero m\'as lleno que Punta
Mogotes en Mar del Plata, as\'i que seguimos a la Villa. Fue un poco cansador.

Mucha gente en el camino nos tira buena onda, algunos no lo pueden creer. Nos
cruzamos un paral\'itico en una especie de triciclo movido por sus brazos.
\textexclamdown Con una camioneta de apoyo recorr\'ia los 7 lagos! Hay que tener
fuerza de voluntad para hacer eso. Y ganas de vivir. Un grande.

\textexclamdown Olvidaba de contar sobre una subida largu\'isima y terrible con
viento en contra! Pero eso no es nada, imag\'inennos transpirad\'isimos, con
cada auto ech\'andonos mucha tierra. Pero les digo que no es nada, ahora
imag\'inennos subiendo a 4 km/h en el cambio 1-1, rodeados por \emph{siete}
t\'abanos que se met\'ian en los ojos, orejas, nariz, boca, picaban las piernas,
y qu\'e se yo que m\'as. \textexclamdown El mal humor de Ezequiel fue lo m\'as
divertido que viv\'i, en ese tramo! Fue realmente insoportable pero yo me
divert\'i mucho con esos molestos zumbidos. En la bajada los perdimos, y pudimos
seguir viaje en paz.

En Villa la Angostura conocimos el Parque Nacional Los Arrayanes. A eso no lo
puedo describir ni en video, ni en foto, ni en persona, ni menos que menos
escribiendo. Pero probemos. Eran 12~km a trav\'es de Alerces, Coihues y otras
yerbas, de un camino lleno de subidas y bajadas sin viento, dentro de un t\'unel
de vegetaci\'on del ancho promedio de dos personas y rodeados por el lago Nahuel
Huapi. Ah\'i fuimos r\'apido de verdad, por lo que era el camino. Como las
alforjas estaban en el campamento no hab\'ia problemas en las curvas. Esos 12~km
se nos tornaron 3~km, primera vez de todo el viaje en que lo que hicimos
pareci\'o menos. Recorrimos el parque, incre\'ibles los lisos, fr\'ios y
anaranjados Arrayanes, y volvimos.

Esto merece m\'as que mil palabras; imaginen, simplemente, el para\'iso terrenal
para el cicloturista. Despu\'es de esto no espero nada del Bah\'ia Manzano
recomendado, porque no puedo imaginar algo mejor. El camino a veces estaba
escalonado, o lleno de ra\'ices, as\'i que destrozamos las llantas, pero
de verdad que no tiene precio recorrerlo. Ah\'i s\'i que planeo volver diez
veces por a\~no, m\'as all\'a de los 7 Lagos.

Ya de noche, empezamos a pedalear en busca de alg\'un campamento saliendo de la
Villa. Ezequiel insist\'ia en seguir a la luz de la luna hasta Bariloche, pero
entre cansado e inseguro en la ruta, lo convenc\'i de parar. Encontramos un
campamento pasando Bah\'ia Manzano. Ma\~nana recorreremos por aqu\'i
descansando, ya que pasado viajamos 70~km hasta Bariloche, y lo haremos en un
solo d\'ia. All\'a saldremos de joda una noche, recorreremos el Circuito Chico y
dem\'as yerbas barilochenses. Eso es todo, creo. No\ldots\ qu\'e va a ser todo.
\textexclamdown Qu\'e lindo recordar!

Apenas vimos el Nahuel Huapi desde la ruta nos volvimos locos de la emoci\'on,
pero tambi\'en de la sed y garganta seca. No s\'e estimar pero a m\'as de quince
metros de altura sobre el lago est\'abamos seguro. Empezamos a deslizar tipo
``culi-pat\'in'' por la tierra, ensuciamos todo, pero llegamos a las rocas
cercanas al lago. Fue adrenal\'inico pero vali\'o la pena: ahora pod\'iamos
tomar agua fresca y ve\'iamos pececitos chiquitos recorriendo el lago. La
cuesti\'on era ahora subir, y, obvio, por esa tierra resbaladiza no lo
lograr\'iamos. As\'i que rodeamos un poco el lago, subimos a trav\'es de rocas
un poco flojas, y encontramos plantas. Miramos arriba y ahora era acantilado,
pero siguiendo un poquito m\'as pudimos subir a trav\'es de tierra,
agarr\'andonos de las plantas. Llegamos arriba despu\'es de mucho trepar.
\textexclamdown Casi terminamos el agua que bajamos a buscar en ese viaje cuesta
arriba!

Madre, no sab\'es lo que valieron tus alfajores ``Capit\'an del Espacio'' ac\'a,
en el Sur. Otro comentario sin relaci\'on: de vez en cuando cruzamos un cami\'on
internacional y me dan ganas de manejarlo.

Agarren un mapa porque sino no entender\'an ni la mitad de lo que habl\'e.
\textexclamdown Con decirles que yo aprend\'i bien el mapa recorri\'endolo!

Bueno, si quer\'ian que escribiera, \textexclamdown lo lograron!

Un fuerte abrazo a todos,

Tute.

\section{De Villa la Angostura a\ldots\ \textexclamdown Bariloche!}

Hoy salimos del camping de Bah\'ia Manzano en Villa la Angostura, a recorrer la
bah\'ia. Me hizo acordar mucho a Caril\'o, con la diferencia de que si
gir\'abamos la cabeza a alg\'un lado siempre se ve\'ia el lago. Salimos para
Bariloche sin saber si lleg\'abamos al anochecer o a la madrugada siguiente,
pero el viento quiso que lleg\'aramos a las 8 pm, bien de d\'ia ac\'a en el Sur.
Un viento a favor infartante, las subidas eran planicies, las planicies bajadas,
y las bajadas eran euforia.

Mucha gente saluda a los ciclistas para levantar los \'animos, y lo logran.
Hicimos en un d\'ia 70~km casi, llegamos bien. Nos sacamos fotos en los
\'ultimos carteles de distancias, y calculamos unos 330~km recorridos hasta
ahora, sin contar las vueltas que se avecinan por Bariloche.

Aunque no lo crean\ldots\ \textexclamdown paramos en el mism\'isimo hotel de
nuestro viaje de 5$^\circ$! Nuestra habitaci\'on est\'a justo en frente de donde
dorm\'ia la mitad de nuestro grupo de amigos. \textexclamdown Lindos recuerdos
que hay al volver ah\'i! Pagamos m\'as que siempre, pero se justifica f\'acil:
camas, \textexclamdown s\'abanas!, ducha, inodoro, \textexclamdown bidet!, TV,
alfombra y\ldots\ \textexclamdown jab\'on! \textexclamdown Jab\'on, despu\'es de
ocho d\'ias! Fue incre\'ible. Diez minutos de agua caliente sin que te joda si
alguien abre la canilla fr\'ia y te queme, o la caliente y te congele. Me
enjabon\'e hasta las patillas, y antes de enjuagarme me frot\'e; sobre la
ba\~nera blanca se iba tierra, y tierra, y tierra\ldots\ \textexclamdown
Incre\'ible la cantidad que tra\'ia! \textexclamdown Un asco!

Ma\~nana voy al lavadero, porque la (casi \'unica) remera no solo tiene olor a
humo, transpiraciones o al lago; el jab\'on blanco tambi\'en hace lo suyo. Y
tiene un color marr\'on que mata. Las medias la van a pasar bien en la
lavander\'ia, tambi\'en.

Ahora estamos en un cyber despu\'es de comer una pizzita; ma\~nana nos iremos
del hotel a un campamento libre para no secarnos de plata, y esa ser\'a la noche
de jodas en Bariloche porque lo vamos a encontrar a Eugenio, otro amigo que anda
por ac\'a.

\textexclamdown Hoy por fin voy a dormir sin pasar fr\'io en alg\'un momento de
la noche! Ma\~nana recorreremos algunas calles de Bariloche y el circuito chico.

Ma, llego a ``Mardel'' el 22 a la noche. Nos vemos;

\textexclamdown Tute!

\section{\textexclamdown Lo hicimos! \textquestiondown Y ahora?}

Hicimos el circuito chico, fue entretenido. Vimos de diversos puntos el lago, y
nos cruzamos en el camino --por primera vez de tres-- una Porsche Cayenne Turbo.

Ibamos a 40~km/h en una bajada, cuando me di cuenta de que en ese punto nos
hab\'ian sacado la foto en el viaje de 5$^\circ$ a\~no, con el hotel ``Llao
Llao'' de fondo. Clav\'e los frenos y posamos para otra foto. Subimos hasta la
puerta del hotel, volvimos a bajar, recorrimos la playita frente al lago, y
seguimos. Encontramos un lugarcito entre \'arboles para parar que ten\'ia un
cartel que invitaba a una caminata de 45' hasta el lago Escondido (el de
Bariloche). Dejamos las bicis a la vera del camino (locura) y empezamos.

Nos sacamos una foto con Arrayanes en un peque\~no bosque que encontramos (nos
quitamos las ganas: \textexclamdown no ten\'iamos esa foto del Bosque!) y
caminamos por un camino rodeado de ca\~nas que hac\'ian de t\'unel. Lindo
sendero, con el Nahuel Huapi a veces borde\'andonos. Caminando y corriendo
llegamos de nuevo hasta la ruta del Circuito Chico, \textexclamdown qu\'e
locura! El Circuito Chico es una gran curva, y lo que hab\'iamos hecho con la
caminata era cortar camino por una recta. O sea que hubi\'esemos llegado
tranquilamente al Lago Escondido de haber seguido pedaleando, como luego
har\'iamos. Claro que igual vali\'o la pena: esa caminata fue divina. Volvimos a
donde estaban las bicis, y seguimos.

Despu\'es de una curva y contracurva por la ruta que agarramos muy r\'apido,
llegamos al Escondido, un peque\~no y pintoresco lago usado por nadadores. Un
lindo muelle, y mucha gente. Seguimos el circuito y pasamos por una cascada,
donde paramos a tomar un chocolate y a sacarnos una foto: era un lugar que
tambi\'en visitamos en 5$^\circ$. Continuamos camino, hasta darnos cuenta que
lleg\'abamos al cementerio del monta\~n\'es. Entramos. Lindo lugar: 5' de
caminata cuesta arriba para llegar al pie de una monta\~na, desde donde se
ve\'ia el Lago y varias cruces, historias y cosas. Muy interesante. Seguimos,
paramos en Colonia Suiza donde Eze compr\'o un licor para regalo (con Euge no
nos encontramos de noche, hablando del tema), miramos el paisaje, y continuamos
cuesta abajo entre curvas. A 40~km/h pasamos una curva de 90$^\circ$ que termina
en puente. \textexclamdown Todo era as\'i! Qu\'e lindo. Hab\'ia que frenar
(\textexclamdown en bici!) para no pasarnos en las curvas. Me parece imposible:
en la bici, aparte de poder doblar mucho, nunca se anda tan r\'apido. Ah\'i
s\'i.

Luego de una bajada hab\'ia una curva de 180$^\circ$. Se me empez\'o a cerrar
m\'as de lo que pod\'ia doblar, y tuve que frenar para no caer a la banquina.
Segu\'i hasta que encontr\'e un parador donde esperar a Eze. Com\'i una rica
mitad de cereza que hab\'ia tirada en el camino (la otra mitad estaba podrida,
seguro alguien la tir\'o por la ventanilla), y segu\'i esperando.
\textexclamdown Cuando lleg\'o me cont\'o que en esa cerrada curva casi choca
contra un auto! El auto segu\'ia derecho para tomar un desv\'io, y \'el lo
esquiv\'o y por eso se cay\'o, cortando el cable del freno trasero.
\textexclamdown Y no tuvo m\'as que algunos golpes! Menos mal, era maniobra
peligrosa.

Seguimos pasando d\'ias en Bariloche, recorriendo, leyendo, conociendo,
descansando. Nos adelantamos al pasaje de vuelta. Al camping hab\'ia tres o
cuatro cuadras de subida empinada. \textexclamdown Un d\'ia la logramos subir
sin bajarnos de la bici! A la bajada era un verdadero peligro, porque no se
pod\'ia controlar muy bien la bici entre la pendiente y los serruchos. Adem\'as,
terminaba en curva cerrada, y si sub\'ia justo alguien lo pod\'iamos llevar
puesto. \textexclamdown Pero nunca ocurri\'o, por suerte! Cu\'anta suerte
tuvimos este viaje.

Terminamos en la Terminal, luego de embalar las bicis. Eze se fue en un cole 2
horas antes que yo (un viaje desastrozo: \textexclamdown se le sali\'o una rueda
y tuvo 5 horas de retraso!). Yo me fui dos horas despu\'es por ``Via
Bariloche'', nada mejor. \textexclamdown Con los \$0,10 que ten\'ia pretend\'ian
que le dejara propina al maletero para que subiera mi bici, sabiendo que no
hab\'ia pagado sobrepeso! Imposible, pero la subi\'o igual y ni acept\'o los 10
centavos, muy buena onda. El azafato del cole hab\'ia visto eso, y me dej\'o
doble raci\'on de cena caliente m\'as la entrada fr\'ia. Qu\'e atenci\'on.
\textexclamdown Estaba sobre un colch\'on, despu\'es de 15 d\'ias durmiendo
sobre tierra! No lo pod\'ia creer. Dorm\'i y com\'i profundamente.

Me baj\'e en Mar del Plata, y fin del excelente viaje.
