\subsection*{Jueves 15 de Septiembre}

\textexclamdown Hola, familia y amigos! \textquestiondown C'omo andan? Yo les
cuento que muy bien (no me gusta andar con novedades sobre estos temas).

Ayer andaba paseando por la joyer'ia, y una espa\~nola observ'o mi remera de
Asturias, no tanto por espa\~nola como por ser ella asturiana.

\subparagraph{}\label{ssub:deQueParte} --- \textquestiondown De qu'e parte?\\
--- De Gij'on.\\ \hangindent=1cm

Cuando le cont'e que un mes antes visit'e ah'i familiares concluy'o: ``es que el
mundo es un pa\~nuelo''. Y hablando de remeras, \textexclamdown me tengo que
llevar una alemana! No compr'e nada por ac'a, aparte de doscientos panes por el
sur. Tambi'en conoc'i a una maratonista de 29 a\~nos tan linda y simp'atica que
me interesaba m'as eso que el hecho de que corra maratones. Fing'ia inter'es
mientras le mostraba las fotos. Cuando llegu'e a la parte de Bernd Albrecht
concluy'o: ``\textexclamdown pero ese tipo est'a loco!'' No lo niego: en varias
fotos que me mostr'o sus pies parec'ian de pl'astico. Menos mal que decid'i
volver a Argentina, porque le propon'ia casamiento a esta chica sino.

\textexclamdown As'i es! Con mam'a contando los planes familiares para su
cumple, y los planes tambi'en muy familiares para la despedida de los novios
(que no tuve en cuenta a la hora de calcular la vuelta), entend'i que son cosas
muy lindas para perderme, as'i que ma\~nana voy a ver si Axel me puede cambiar
el pasaje. Lo pens'e mucho antes, claro. Mucho. Entre los pensamientos entr'o el
que me queda tiempo para conocer el Sal'on de Frankfurt, elemental. Sin embargo
me pierdo el museo de Wolfsburg y tiro la posibilidad de bucear. Pero creo que
vale la pena.

Primero le'i los mails de las cadenas. Despu'es me fui a ense\~narle espa\~nol a
Amira pero no estaba, as'i que la esper'e cont'andole el viaje a la simp'atica
madre. Ella me cont'o que en Austria anduvieron por partes no tan monta\~nosas y
en granjas de producci'on de vinos, no me perd'i demasiado entonces. Y con las
esperanzas de para'iso que ten'ia, me desilusionar'ia; todo se fue dando del
mejor modo posible. Luego, y notando que Amira no volver'ia (no saben como odia
al profe por el solo hecho de ser profe, si hablo en ingl'es entonces
sonreir'a), volv'i a casa cuando empezaba a llover.

Por merendar como ternero tuve que esperar a digerir antes de salir a correr a
los bosques. Com'i un pan justito antes de salir, \textexclamdown no me dej'o en
paz en todo el recorrido! Qu'e esfuerzo es respirar cuando el est'omago {\sl
tambi'en} quiere hacerlo, lo 'unico malo de esta salida.

Mirando el verde clar'isimo de los 'arboles mientras todo estaba oscuro por la
constante lluvia, patinaba con piedras tropezaba con ramas, \textexclamdown as'i
de natural es! Anduve siempre por caminitos en subida o bajada, muy accidentados
y a veces m'as angostos que yo, as'i que me ten'ia que agachar para no empaparme
con las ramas que chocaba. Igualmente de la lluvia no me proteg'i demasiado
porque daba menos fr'io.

Mientras corr'ia por esta belleza de bosques (eso s'i que lo voy a extra\~nar,
qu'e placer tener a diez cuadras este para'iso natural) pensaba en volver a
Sudam'erica, as'i que ya en casa, y luego de elongar no mucho pero ba\~narme
bastante, les cont'e la idea a los padres postizos. Despu'es llam'e a casa,
sucursal Pergamino:

\subparagraph{}\label{ssub:holaVieja} --- Hola Ma {\sl (``\textexclamdown
hoooooooola querido!'')}, \textquestiondown por qu'e ruta viajan a Necochea?\\
--- Por la ruta ``$x$'', \textquestiondown porqu'e?\\ --- Para que vayan por la
2, y me pasan a buscar por Ezeiza. \textquestiondown Les parece?\\
\hangindent=1cm

Primero pens'o que era broma, y no me daba bolilla. Despu'es se emocion'o, y
tampoco me daba bolilla. \textexclamdown Pero yo no sab'ia, as'i que no
entend'ia porqu'e el silencio! Confirmaron el cambio de ruta, y a cambiar mi
pasaje.

\textquestiondown Saben qu'e? El 5$^\circ$ viaje (centro de Espa\~na, Asturias,
M'alaga, Alemania, {\sl Alemania en bici}) fue tan aventurero por as'i decirlo,
y tan bueno y grande, tan lleno de cosas nuevas, que ya no pude
disfrutar de la vida ``normal'' (aunque seguro con sorpresas, como siempre) que
me espera en G\"ottingen, pensando en los eventos argentinos. Lo 'unico que me
pierdo --y porque ya se me va llenando la agenda-- es el medio marat'on de
Pergamino, que si bien no entren'e especialmente me animar'ia a inscribirme por
la tranquilidad con que se movi'o mi corazoncito durante el trote de hoy. La
'ultima fue en dos horas y a esta no se siquiera si llego, pero el hecho
de estar ah'i, aunque no corriendo, ya infunde alegr'ia. Es el Domingo 25.

Bueno, \textexclamdown muchas cosas dando vueltas por la cabeza en planes y
organizaciones! \textexclamdown Pensar que en una semana volver'ia si todo va
bien, y no compr'e regalos! Obvio porque no ten'ia plata, pero al menos una
tarjetita, un tarrito de rollo de fotos con tierra, una manzana con granos que
no se pueden comer como encontr'e tantas por mi viaje\ldots\ se van a sorprender
de lo taca\~no. Y no lo digo para que reciban sin esperar, \textexclamdown lo
digo desde la m'as profunda sinceridad! Lo 'unico que compr'e fueron vasitos
para los amigos, promesa en la fiesta de despedida. Son tan chicos que sirven de
dedal, pero tambi'en para adornar, y si los miran con mucho cari\~no, para tomar
tequila.

\textexclamdown Muchas cosas dando vueltas por mi cabeza!

\subsection*{Domingo 19 de Septiembre}

Saqu'e pasaje en tren 12 horas antes de que salga mi avi'on, as'i hago metralla
de fotos en el Sal'on del Autom'ovil de Frankfurt. Les quiero contar mis
'ultimos respiros en G\"ottingen.

Primero, desde que llegu'e me sorprendo de que gente de afuera de la familia me
pregunte con inter'es c'omo me fue. Me alegra mucho; saben, se siente raro
volver de un viaje en bici por Alemania a un pueblo de Alemania, y que vengan
``amigos'' casi reci'en conocidos a preguntar c'omo anduvo el viaje y que les
cuente.

Cuando entr'e en las oficinas de Axel para dar el presente, casi con orgullo les
present'o mi viaje a sus cotrabajadores:

\subparagraph{}\label{ssub:cuantoskm} --- \textquestiondown Cu'antos kil'ometros
fueron?\\ --- Mil cien.\\ --- \textexclamdown Mil-cien kil'ometros!\\
\hangindent=1cm

Axel sonre'ia mientras escuchaba con inter'es los cuentos. Le volv'i a agradecer
tanto la bicicleta.

La primer ma\~nana luego de llegado sal'ia en bici a la joyer'ia, y Felipe
lloraba porque quer'ia venir conmigo. \textexclamdown Creo que ni saben que
hice, pero saben que desaparec'i! Saben, tambi'en, que en Marzo visitan
Argentina. Pablo s'olo pregunt'o cu'ando vuelvo.

\subparagraph{}\label{ssub:cuandoVolves} --- A la tarde.\\ --- Trat'a de volver
aaanteees\ldots\\ \hangindent=1cm

Le contaba esto a Gustavo, sorprendido porque en mi vida jugu'e con ellos o les
hice alg'un d'ia m'as interesante, y 'el, afirmando esto 'ultimo entre risas,
no se sorprend'ia porque los conoce.

Un Viernes que ni idea ten'ia era Viernes estaba por ir a dormir cuando me
invit'o Patrizia, empleada de Axel, a salir. Nunca sal'i en Alemania, y no solo
porque no nombraron la discoteca fue que acept'e. Empezamos en un bar lleno en
que quedaba una mesita casi reservada; si nos qued'abamos no le err'abamos
despu'es me di cuenta. Tom'e el primer fernet despu'es de un tiempo, qu'e bueno
es. No se sorprendi'o ella de que me guste ``{\sl eso}'', \textexclamdown pero
que adem'as lo saboree de tal modo era demasiado! De ah'i seguimos a Savoy, que
me record'o mucho a Specktra de Pergamino no por tama\~no pero por su onda: en
su 'epoca debe haber sido un infierno y todav'ia est'a bueno. Ella dec'ia ``los
viernes no va gente a Savoy, s'olo los tontos como nosotros, que van creyendo
que con los tragos libres hasta las 12 hacen negocio''. Ac'a se sale los
mi'ercoles, en Buenos Aires los Jueves, y creo que nunca le voy a encontrar
l'ogica. \textexclamdown Esta es ciudad casi exclusivamente universitaria! Los
que no estudian trabajan. Pero se sale.

La onda de la disco es igual que en Argentina: la gente, las decoraciones,
\textexclamdown la m'usica! Pasaron tanta m'usica latina que dudaba de estar en
Alemania o en Colombia. Y un poco de m'usica ``negra''. Un ebrio para pedirme
paso charl'o 45 mins, entre los cuales por poco me agradec'ia por visitar su
pa'is. Su poca costumbre con el ingl'es hac'ian menos comprensibles las ideas,
pero eso fue lo 'unico en claro: estaba interesado en hacerme entender que le
gusta la diversidad de culturas. Volvimos temprano a dormir, porque esa ma\~nana
era S'abado y trabajaba.

\subparagraph{}\label{ssub:hoyesSabado} --- \textquestiondown S'abado?
\textquestiondown Pero no es Domingo?\\

Y se volvi'o a re'ir y a envidiarme de lo perdido que ando en estos tiempos de
largas vacaciones. Empezando porque me cont'o en el bar que era Viernes y por
eso hab'ia bastante gente.

Antes que todo esto me llamo mi profesora de alem'an Lissy (que por suerte al
final no se tom'o muy a pecho lo del idioma) para ir a Savoy. Como yo no bailo,
pero adem'as no es que no tengo onda sino que tengo mala onda en los boliches,
le dije que no. \textexclamdown Por eso se calent'o al encontrarme en una barra!
Al menos era verdad: no bailaba y estaba porque ven'ia de un bar. Qu'e grandes,
Axel y Gustav: ellos me presentaron a medio G\"ottingen el primer d'ia que lo
pis'e.\\

Hoy (Domingo a la ma\~nana) fui a nadar con Lissy. \textexclamdown Primero
desayun'e, obviamente, o qu'e esperaban! Hab'ia una niebla que no se ve'ia al
vecino, pero Ina estaba casi segura del d'ia soleado. \textexclamdown Yo cre'i
que en este tiempo se hab'ia vuelto loca! Incre'ible que tuviera raz'on: al
levantarse la niebla el sol iluminaba desde el azul'isimo cielo. Estuvo bien el
nado, como siempre. Estaba re-fr'io pero el agua templada. Afuera todav'ia
hab'ia niebla y siempre es hermos'isimo verse en esa pile externa que larga
vapor.

Despu'es cac'e la bici (limpita y como nueva) y fui a la torre de Bismark
arriba en los bosques, para tener una buena vista a 360$^\circ$ de la ciudad.
Hermos'isimo todo, me encantan los puntos panor'amicos. Se ve'ia todo
G\"ottingen, \textexclamdown los bosques!, algunos pueblitos adyacentes, rutas y
caminos alej'andose, y la turbina e'olica m'as grande, todo como en una buena
maqueta. De ah'i a bicicletear por barrios no conocidos de G\"ottingen
(hermos'isimos como la mayor'ia de la ciudad) y a un pueblo muy cercano que
tampoco conoc'ia y quise conocer. \textexclamdown Ten'ia el nombre m'as raro de
todos los pueblos que visit'e! Ni siquiera sonaba alem'an, no lo memoric'e. No
ten'ia nada interesante y volv'i a comer el asado que hac'ian, el pollo de
Gustavo no tiene precio.

Esa tarde iba a ir a Hanstein de nuevo, por el otro castillo que me falta para
ver la ``ca'ida del diablo'', un mirador, pero estaba cansad'isimo. Claro, nad'e
sin pensar que eso cansa (la espalda no son las piernas), despu'es anduve en
bici queriendo andar r'apido pero poco, cre'i que no cansaba. \textexclamdown
Llegu'e y de suerte no me dorm'i en la mesa! De invitados: una pareja amiga
de la familia. As'i que espero no perderme la ca'ida del diablo antes de la
vuelta, ver'e, y espero que el buen tiempo siga acompa\~nando.

No paro de comer y sigo en 70 kg clavad'isimos, llegu'e del viaje, y 70. Yo no
se como hago pero es incre'ible. Siesta y a escribir.

Compr'e un vino ``Barolo'' para dar a Axel junto con su bici, y comet'i el error
de preguntarle a Gustavo qu'e quiere que le regale a 'el. Obvio ``nada'', pero
yo le dije que necesitaba irme sintiendo que le dejo algo en la casa. Un vino no
porque se acaba, algo que use y le guste y recuerde. Entonces contest'o: ``Voy a
pensar c'omo te puedo dejar ese sentimiento''. \textexclamdown Le voy a regalar
algo {\sl bien} feo y va a servir para lo que quiero!

Como ven cay'o la ficha de que me voy. \textexclamdown Quiero ver todo lo que
en los mapas no conozco y que alguna vez estuvo en planes!

\subsection*{Lunes 20 de Septiembre}

{\sl Qu'e bueno} que hoy no ten'ia computadoras en lo de Axel para bajar mis
fotos a {\small CD}s. Compr'e una azucarera a Gustavo despu'es de
criticarle la suya en el desayuno, busqu'e alguna remera alemana que en
Argentina voy a lamentar no haber encontrado, y com'i un mont'on de pan con un
montonazo de mermelada casera antes de salir, a las 11:30am, en la bici hacia el
sur de G\"ottingen. Me faltaba un castillo, {\sl Ludwigstein}, y un mirador, la
``ca'ida del diablo'' o {\sl Teufelskanzel}. Me parece que estos castillos
pertenecieron a Ludwig I, pero la verdad no me interes'e. Tambi'en desde el uno
puede verse al otro.

Luego de un buen viaje a la ida, llegu'e r'apido a Bornhagen, el pueblo del
castillo Hanstein. El camino es hermos'isimo, primero llano pero despu'es se va
metiendo en colinas cada vez m'as altas, por una ruta angosta y casi sin autos,
lind'isimos pueblos, hasta que se llega a ver en la punta de un cerro al viejo
castillo. Sol radiante y aviones para tirar para abajo. Al llegar a la entrada
del bosque com'i dos duraznos mirando al castillo ya conocido, antes de meterme
en el oscuro camino para llegar a la Ca'ida del Diablo. Mientras miraba un mapa
tur'istico que mostraba las principales atracciones de la regi'on, not'e que, un
poco por inter'es y otro poco por perdido, conoc'i a casi todas.

Esto era muy alto, a pocas cuadras del primer castillo; ya tuve la primer gran
subida del d'ia. Me inund'e de bosques con la bici, todav'ia subiendo muy
despacito, por un camino pedregoso e irregular que no permit'ia aburrirse. En un
momento se hizo un claro de 'arboles y pude ver a las colinas lejanas, estaba
alt'isimo.

En una bifurcaci'on sin demarcar tom'e el camino de la izquierda, y empez'o a
bajar. Me sent'ia perdido porque se supone que el mirador est'e arriba de todo.
Estos metros volv'i a sentir lo que tantas veces: \textexclamdown el estar
andando por el camino equivocado y puteando hasta a las ramas! Subidas y bajadas
de por medio, encontr'e una pareja que caminaban con su perro cerca de una
caba\~na. Les pregunt'e por el mirador se\~nalando el mapa, y me dijeron que
``est'a ah'i'', detr'as de la caba\~na. Sub'i sospechando que mi error con el
camino fuera cierto pero ah'i estaba: una gran roca bajo una ventana formada por
la rara ausencia de 'arboles. Sub'i, y era incre'ible. Desde
esa altura ver colinas m'as bajas que uno es una sensaci'on dif'icil de
explicar. Detr'as de la primer hilera de colinas, m'as colinas alej'andose y
perdi'endose en la bruma. Adem'as pensaba en que llegu'e ah'i transpirando:
ahora pod'ia bajar usando la energ'ia de la gravedad que hab'ia ganado. Al fondo
las monta\~nas se ven color azul oscuro por la fina niebla que las difuminaba.

No s'olo se ve'ia eso porque adem'as estaban el color negro de bosques sobre las
sierras como las manchas de una vaca, a la derecha toda esta colina con bosques
y 'arboles verde oscuro que bajan hasta un r'io, que justo ah'i describ'ia una
``{\small U}'' para volver a alejarse. A la izquierda de la {\small U}, un
peque\~no pueblo: {\sl Lindewerra}. Dentro de la {\small U} de agua (que no es
muy grande) cuatro colores diferentes para los diferentes cultivos del campo.
Pintoresco el para'iso.

Volv'i a la bici a comer el tercer durazno y tomar poca agua (el fr'io no
permiti'o transpirar demasiado), y en vez de bajar al r'io por el camino para
bicis, lo hice por el pedestre.

Es un desperdicio bajar a 5~km/h y frenando todo lo que uno subi'o, pero fue
{\sl tan} placentero. Imaginen los caminos angostos de los bosques. Yo cre'i
que en el piso hab'ia aserr'in por las le\~nas de la caba\~na pero era un gran
colch'on de hojas y ramas; a algunos 'arboles ya les lleg'o el oto\~no. Aparte
de esto que tornaba poco firme la superficie hab'ia piedras grandes, bajadas
empinadas, al principio el precipicio a mi derecha que en las condiciones no me
dejaba de lo mas c'omodo, y las ruedas casi siempre yendo de costado por querer,
sin lograrlo, subirse a piedras o ramas; nunca derecho como en una calle.
\textexclamdown As'i de divertido! En un momento el camino se divid'ia en tres
bajadas: a la derecha para bajar de un salto, al medio para bajar por algo
similar a una empinada escalera, y a la izquierda para bajar caminando con
cuidado. Tom'e el de la izquierda, pero patin'e y ca'i al medio; tuve la suerte
de chocar contra la pared del tercer camino, que me fren'o. De no ser por la
mala suspensi'on y por tanto baj'isima velocidad hubiese estado al nivel del
Bosque de Arrayanes, hubiese sido de lo mejor que viv'i en la bicicleta. Y ahora
que lo recuerdo entiendo el dolor en el codo izquierdo, mientras volv'ia y me
preguntaba: ``\textquestiondown qu'e pasa?'' Porque cuando llegu'e a la luz y a
un camino derecho (y a diferencia de la bajada llegando al {\sl SeeburgSee})
liber'e los frenos y mir'e a lo lejos, ya no a donde pisaran las ruedas. Pero
aqu'i las canaletas tienen alrededor de 7 metros de profundidad, \textexclamdown
as'i que me sorprendi'o la primera que me ense\~n'o por fin a bailar!

Cruc'e el r'io, y luego de tomar el sentido opuesto al segundo castillo, volv'i
y llegu'e al camino correcto. Pas'e por un cementerio de guerra, siempre miro
las fechas y no se para qu'e, porque al calcular las edades, siempre me
sorprenden. Pero todav'ia no encontr'e ninguna muerte en el ``d'ia D'';
no los enterrar'ian por esta zona, pero tampoco vi en monolitos de iglesias.

El mapa indicaba punto panor'amico para el castillo pero no lo pude apreciar
demasiado a trav'es de las ventanitas de su torre. Del a\~no 1600, reconstruido
varias veces seg'un las planillas informativas, lo usa la juventud desde hace un
buen tiempo. Por la vestimenta puede tratarse de boy-scouts, pero no estoy
seguro si lo son. Durante la 'epoca fue usado por la juventud hitleriana, ver a
esos ni\~nos saludando en la forma nazi conmueve mucho. Hermoso castillo, por su
construcci'on y vista.

La bajada fue emocionante como supon'ia mientras la sub'ia, mirado los
paisajes que entonces ni pispear'ia. En una curva se me fue la mano con el freno
trasero, y la cola quiso seguir derecho, esta vez en seco as'i que en el pasto
desaceler'e y pude volver al camino. \textexclamdown Incre'ible! De terminar en
ca'ida la rodilla quemada me acompa\~nar'ia algunas semanas, por suerte no pas'o
nada.

Vi muchos aviones cruzando el cielo despejado. Pas'e por una ruta en
construcci'on, mir'e el mapa e indicaba castillo. Recorr'i el lugar con la vista
pero no encontr'e ninguno, cre'ia estar perdido pero despu'es encontr'e la
construcci'on antigua, sobre una suave colina. \textexclamdown Tres castillos
distintos a 30~km desde G\"ottingen!

Par'e en la ``estancia que no es estancia'' (Besenhausen), pero los Lunes y
Martes cierran. \textexclamdown Qu'e mala suerte! El chocolate que mi fr'ia
garganta y yo ansi'abamos tendr'ia que esperar. Salud'e a la familia que cuidaba
el lugar y me invitaron a quedarme a descansar si lo deseaba. Dormitaba echado
en el pasto mirando al cielo, calentito al sol despu'es del fr'io viento. Alg'un
ganso tuvo gansitos y andaban todos los pollitos amarillos de aqu'i para all'a.
Hab'ia un pato con cabeza de color verde oscura, casi negra a la sombra, pero al
sol adquir'ia un tono tan fuerte que no parec'ia natural. Com'i un pan que me
quiso robar un viejo y simp'atico labrador (``la granja de Orson'' parece esto),
tom'e agua y continu'e el pedaleo.

Saqu'e fotos a G\"ottingen por primera vez desde la ruta, a modo de despedida.
Llegu'e a casa, hice cereales con leche, y se acerc'o un amigo de Felipe que se
qued'o mir'andome fijo con esa cara de viejo labrador que me invitaba a
invitarle. \textexclamdown El sinverg\"uenza se qued'o hasta que terminamos la
compotera! Al menos me ense\~n'o una palabra, que por su sonriente expresi'on
defin'i como rico: {\sl ``Lecker!''} Y no se equivocaba.

Un buen ba\~no y a leer mails. \textexclamdown Nunca me llegaron tantos, entre
esa despedida de solteros y casamiento!

\subsection*{Martes, 21 de Septiembre}

Ya me voy, el calendario sabe desde hace unos d'ias pero yo me enter'e ahora.
Empec'e por las oficinas, donde guard'e todos mis archivos y desinstal'e
programas de ``mi'' {\small PC}. Ah'i cayo la ficha. Es verdad lo que dice el
pasaje, \textexclamdown me voy!

Luego, a tomar un vaso grande raro en un restauran italiano con leche y caf'e y
no se que m'as, que estaba buen'isimo (un {\sl ``latte macchiato''}), en una
galer'ia al sol de la peatonal ``linda'' de G\"ottingen. Con mi ``amiga'' Lissy,
que por c'omo se despidi'o parece que soy su amigo, sin comillas. Cre'i que las
despedidas no molestaban pero me dej'o de cama con esas despedidas del estilo
``qu'e bueno que nos conocimos, alguna vez nos vamos a reencontrar\ldots'' No se
si evitarlas es de poco caballero, de cag'on u otro modo de decir el (no tan)
simple ``adi'os''. \textexclamdown Me encari\~n'e demasiado con este lejano
pa'is me parece! No es tan f'acil como esperaba preparar la mochila y subir al
tren.

De ah'i a comer Sushi con Gustav, riqu'isimo. Nos sentamos en una barra que
rodea al cocinero, tiene un canalcito de agua por el que navegan botecitos de
madera con los alimentos, muy bueno.

Despu'es y casi por obligaci'on fui a comprar la remera de la universidad de
G\"ottingen. Qu'e lindo clima hay en las universidades, todos disfrutaban en el
campo el gran d'ia de sol que hoy es. Todos van en bici, no se imaginan las
cientos de bicicletas que hab'ia. Y todos con uno o dos libros bajo el brazo
pero en tiempo relajado, charlando entre amigos y bien vestidos. Me encantan las
universidades. \textexclamdown Me sent'i vagazo entrando de ``sport'' y para
llevar una remera de recuerdo!

De ah'i a devolver la bici, el momento m'as dram'atico del viaje, pero por
suerte en la casa de Axel no hab'ia nadie. As'i que aqu'i est'a, la hora de mi
vuelta no parece haber llegado sin haberla entregado agradeciendo. Pensar que me
la dio apenas llegu'e por dos semanas\ldots\ \textexclamdown y la us'e hasta mi
'ultimo d'ia de viaje!

Termin'e de leer la ``Se\~norita de Tacna'' y me encant'o, leer obras teatrales
acent'ua ese sentimiento de que uno se imagina lo que podr'ia estar viendo,
agrega placer a la lectura. Aunque por otro lado, la detallada descripci'on de
la escenograf'ia no deja demasiado trabajo a la imaginaci'on.

Mir'e el reloj, quer'ia que se pasen las horas de este diazo porque decid'i no
salir a correr para que los bolsos no sufran con mis sales. ``\textquestiondown
Las 13 todav'ia? \textexclamdown La hora no pasa nunca!'' pens'e,
\textexclamdown pero se hab'ia parado mi reloj! Se dio el lujo de acabar su pila
justito antes de tomar un tren a las 7:18, un avi'on a las 19:25 y otro a las
23:22. Me llevo el despertador pl'astico y grande en el bolsillo.

Pas'e el cuentakil'ometros a la bici de Gustavo. Luego los chicos me regalaron,
cada uno, un playmobil diferente. Un caballero sobre caballo se lleva todos los
premios; el otro, m'as pirata que tambi'en est'a bueno pero se lleva el premio
de plata.

Cena de las buenas, pollo asado por Ina en la asaderita el'ectrica integrada a
la mesada, jam'on crudo y un vino. Luego un rezo en conjunto que hizo temblar la
tierra. \textquestiondown Se sinti'o desde all'a? Muy lindo. Y a cerrar los
bolsos (eso s'i se guarda para el 'ultimo minuto) y mandar el mail para guardar
todos los archivos en un disco. \textexclamdown Qu'e mudanza!

Flor de viaje. Hoy mientras dejaba al tiempo seguir su curso me puse a ojear un
atlas del mundo (del National Geographic, en cuanto lo encuentre por Argentina
lo llevo a casa) y no par'e de pensar en viajes. \textexclamdown Era un tema
abierto desde que no se qu'e so\~nar luego de viajar a Ushuaia este a\~no! La
otra punta de Argentina, empezando en la Quiaca. Por supuesto, s'olo sue\~nos
que nunca deben faltar, porque despu'es de esto y sin saber qu'e me deparar'a el
destino pensar en viajes es de locos.

Todav'ia no me agarr'o la locura del ``me vuelvo''. Ya me voy a empezar a
arrancar los pelos en el aeropuerto, o a sonre'ir solo, ya veremos.
\textexclamdown Si es que el Sal'on de Frankfurt no me desconcentra demasiado y
me acuerdo de tomar el tren hasta el avi'on!

\textexclamdown Muchas gracias por su compa\~n'ia! Un beso grande a todos, nos
vemos;

Tute.
